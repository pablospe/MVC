\documentclass{beamer}
\usepackage{multimedia}
\usepackage{color}
\usepackage{amsmath}
\usepackage{amssymb}
\usepackage{tikz}
\usepackage{graphicx}
\usepackage{verbatim}
\usepackage{subfigure}
\usepackage{hyperref}
\newcommand{\lwidth}{.35}
\newcommand{\lcwidth}{.8}
\newcommand{\swidth}{.12}
\newcommand{\mwidth}{.18}
\newcommand{\half}{.5}
\theoremstyle{remark}
\newtheorem{rec}{Recall}
\newtheorem{rem}{Remark}
\renewcommand*{\thesubfigure}{}

\newcommand{\I}{^{-1}}
\newcommand{\set}[2]{\{#1|#2\}}
\newcommand{\topic}[1]{\noindent{\textbf{#1}}}
\newcommand{\bij}{\longleftrightarrow}
\newcommand{\bslash}{\setminus}
\newcommand{\cl}[1]{\overline{#1}}
\newcommand{\seq}{\subseteq}
\newcommand{\ds}{\displaystyle}
\newcommand{\Wlog}{Without loss of generality }
\newcommand{\rp}{$(\Rightarrow)$ }
\newcommand{\lp}{$(\Leftarrow)$ }
\renewcommand{\qedsymbol}{\tiny$\blacksquare$}

\begin{document}
\title{Coordinates for Instant\\Image Cloning}
\author{Julius Elinson}
\date{December 12$^{\text{th}}$, 2011}
\beamertemplatenavigationsymbolsempty
\institute{Harvey Mudd College}
\begin{frame}
\titlepage

\end{frame}
\begin{comment}
\begin{frame}[t]
 \frametitle{Image Composition}
 \pause
\begin{block}{Using a mask}
\begin{itemize}
 \item Cumbersome
 \item Same size as image
 \item Changes with each patch
\end{itemize}
 \pause
\begin{block}{Poisson cloning}
\begin{itemize}
 \item Automates the masking process
 \item Creates a membrane for patch area
 \item \alert<4>{Difficult and expensive to solve}
\end{itemize}
\visible<4>{\centering\emph{Requires solving a system of partial differential equations}}
\end{block}
\end{block}
\end{frame}

\begin{frame}[t]
 \frametitle{Image Composition}
 \vspace{2mm}
 \emph{Method proposed by Farbman et al. in paper from 2009:} 
 \pause
\begin{block}{Mean-value cloning}
\begin{itemize}
 \item Approximates poisson cloning
 \item Nearly identical results
 \item Applies simple ideas of interpolation and averaging
\end{itemize}
\end{block}
\pause
\begin{block}{Extensions}
\begin{itemize}
 \item Video cloning
 \item Image stitching
 \end{itemize}
 \end{block}

\end{frame}
\end{comment}

\begin{frame}[t]
\frametitle{Published Results}

\vfill
\begin{center}
\includegraphics[scale=.28]{polar}
\end{center}
\vspace{3mm}
From Farbman et al. in ``Coordinates for Instant Image Cloning''
\end{frame}

\begin{comment}
\begin{frame}[t]
 \frametitle{Results | Mean-value}
 \vfill
\begin{center}
\includegraphics[scale=.28]{bigpolar}
\end{center}
\end{frame}

\begin{frame}[t]
 \frametitle{Results | Mean-value}
 \vfill
\begin{center}
\includegraphics[scale=.28]{huge}
\end{center}
\end{frame}

\begin{frame}[t]
 \frametitle{Results | Poisson}
 \vfill
\begin{center}
\includegraphics[scale=.29]{poisson}
\end{center}
\end{frame}
\end{comment}


\begin{frame}[t]
 \frametitle{Method}
 
\begin{block}{Pseudocode}
 \begin{center}
\includegraphics[scale=.28]{alg}
\end{center}
\end{block}
\end{frame}


\begin{frame}[t]
 \frametitle{Method}
 
\begin{block}{Preliminary Steps}
\begin{itemize}
 \item Two-window user interface -- source and destination
 \pause
 \item Allow user to specify boundary \pause -- both with discrete points and by continuously tracing 
 \pause
 \item Create a complete boundary by linearly connecting points
\end{itemize}
\end{block}
\pause
\begin{block}{Bresenham's Line Algorithm}
\begin{center}
 \includegraphics[scale=.5]{bresenham.png}
\end{center}
\end{block}
\end{frame}


\begin{frame}[t]
 \frametitle{Method}
 
\begin{block}{Preliminary Steps}
\begin{itemize}
\pause 
\item Compute boundary interior with bounding box
\end{itemize}
\pause
\begin{center}
 \includegraphics[scale=.5]{interior.png}
\end{center}
\pause
\begin{itemize}
 \item Allow user to specify location of cloned patch
 \pause
 \item Map source boundary to destination boundary
 \pause
 \item Perform main algorithm
\end{itemize}
\end{block}
\end{frame}

\begin{frame}[t]
 \frametitle{Method}
 
\begin{block}{Computing mean-value coordinates}
  \begin{center}
\includegraphics[scale=.28]{polygon}
\end{center}
\pause
\vspace{2mm}
\begin{columns}
 \column{.5\textwidth}
 \[
 \lambda_i(x)=\frac{w_i}{\sum_{j=0}^{m-1}w_j}
 \]
 
  \column{.5\textwidth}
  \vspace{-4mm}
  \[
  w_i=\frac{\tan(\alpha_{i-1}/2)+\tan(\alpha_i/2))}{\|p_i-x\|}
  \]
\end{columns}
\end{block}
\end{frame}






\begin{frame}[t]
 \frametitle{Method}
 
\begin{block}{Pseudocode}
\vspace{-2mm}
 \begin{center}
\includegraphics[scale=.24]{alg}
\end{center}
\end{block}
\pause
\vspace{-5mm}
\emph{Essentially the resulting pixel is determined by a weighted average of the \alert{change in intensity of nearby boundary pixels}, where the weights are the mean-value coordinates.}
\end{frame}

\begin{frame}[t]
\frametitle{Optimization} 
\begin{block}{Hierarchical Boundary}
\pause
\begin{center}
\includegraphics[scale=.18]{mesh}
\end{center}
\pause
\begin{itemize}
 \item \small{Granularity for each interior point is determined by proximity, gauged by both distance \& angle between adjacent boundary points.}
 \pause 
 \item \small{Implemented by recursively proceeding through levels of boundaries that get successively finer.}
\end{itemize}

\end{block}
\end{frame}


\begin{frame}[t]
 \frametitle{Results}
\begin{block}{Also using polar bears}
\pause
 \vspace{-2mm}
\begin{figure} 
\begin{center}
 \subfigure{\includegraphics[scale=.4]{beach.jpg}}
 \subfigure{\includegraphics[scale=.15]{mybear.jpg}}
\end{center}
\end{figure}
\pause
\vspace{-20mm}
\begin{center}
 \includegraphics[scale=.15]{results/naivebear.jpg}
\end{center}

\end{block}

\end{frame}

\begin{frame}[t]
 \frametitle{Results}
\begin{block}{Also using polar bears}
 \vspace{-2mm}
\begin{figure} 
\begin{center}
 \subfigure{\includegraphics[scale=.4]{beach.jpg}}
 \subfigure{\includegraphics[scale=.15]{mybear.jpg}}
\end{center}
\end{figure}

\vspace{-20mm}
\begin{center}
 \includegraphics[scale=.15]{results/bingo.jpg}
\end{center}

\end{block}

\end{frame}

\begin{frame}[t]
 \frametitle{Results}
\begin{block}{Close-up}
 
\begin{center}
 \includegraphics[scale=.3]{results/closeresult.png}
\end{center}
\end{block}
\end{frame}

\begin{frame}[t]
\frametitle{Results}
\begin{block}{Theirs close-up}
\begin{center}
\includegraphics[scale=.2]{huge.png} 
\end{center}
\end{block}
\end{frame}

\begin{frame}[t]
 \frametitle{Results}
 
\begin{block}{Too much background blending}
 \begin{figure} 
\begin{center}
\includegraphics[scale=.14]{results/hmc.jpg}
\end{center}
\end{figure}
\end{block}
\end{frame}

\begin{frame}[t]
 \frametitle{Results}
 
\begin{block}{Too much background blending}
 \begin{figure} 
\begin{center}
\includegraphics[scale=.14]{results/polar.jpg}
\end{center}
\end{figure}
\end{block}
\end{frame}

\begin{frame}[t]
 \frametitle{Results}
\begin{block}{Too much background blending}
 \begin{figure} 
\begin{center}
\includegraphics[scale=.28]{results/badbear.png}
\end{center}
\end{figure}
\end{block}
\end{frame}

\begin{frame}[t]
\frametitle{Video Cloning}
\begin{block}{Batch jobs}
\begin{itemize}
\pause
\item User provides patch boundary using representative still image
\pause
 \item Clone over two sequences of images, while the patch location is invariant
 \pause
 \item Weighted average of time-decaying interpolant value for smoothing
 \pause
 \item Example
\end{itemize}
\end{block}
\end{frame}

\end{document}